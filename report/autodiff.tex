\documentclass[11pt]{article}
\usepackage{tikz}
\usepackage[height=220mm, width=170mm]{geometry}
%% \usetikzlibrary{calc}

\usepackage{graphicx}
\usepackage{fancyhdr}
\usepackage{xcolor}
\usepackage{listings}
\usepackage{multicol}

\usepackage{amsmath}
%% \usepackage{algorithm}
%% \usepackage[noend]{algpseudocode}

\setlength{\headheight}{15pt}

\pagestyle{fancy}
\lhead{Assignment 2}
\rhead{Kevin Mooney -- 11372946}
\fancyfoot{}

\linespread{1.2}


\begin{document}

\vspace*{2mm}
\noindent
{\centering\textbf{\Large Automatic Differentiation}\\}
\vspace*{5mm}

\subsection*{Compiling \& Running - \texttt{make}}
I used automake as recommended, so to compile and run it should hopefully be as simple as
\begin{itemize}
\item[\$] ./configure
\item[\$] make
\item[\$] \texttt{./autodiff -t 10000000 < ./test/test1.in}  
\end{itemize}  
All source can be found in ./src and some test input files can be found in ./test. The following section describes the interface

\subsection*{Interface}

The interface goes as follow, when you run autodiff without a file you will be prompted with the following screen
\begin{lstlisting}[frame=single]
To start our function is f_0(x) = f_1(x)
Pick f_1(x)

The options are:
0 - As is          1 - polynomial     2 - trigonometric
3 - hyperbolic     4 - exponential    5 - logarithmic 
The composers are:
6 - addition       7 - product     
8 - quotient       9 - power 
\end{lstlisting}

To construct any function we work from the outside in, so as to construct test function 3, $h(x)$, we would take the following steps
\begin{itemize}
\item $h(x) = h_1(x)/h_2(x)$ \dots quotient
\item $h_1(x) = exp(h_3(x))$ \dots exponential
\item $h_3(x) = x^{2.5}$ \dots power \dots As is
\item $h_2(x) = h_4(x) + h_5(x)$ \dots addition
\item $h_4(x) = log(x)$ \dots logarithmic \dots As is
\item $h_5(x) = \coth(x)$ \dots hyperbolic \dots coth \dots As is
\end{itemize}
The commands can be found in ./test/test3.in. The following are the test functions

\subsection*{Test 1}           

\begin{equation}
  f(x) = \sin(x)
\end{equation}

\subsection*{Test 2}           
\begin{equation}
  g(x) = \tan(x^2 + 2x + 1)
\end{equation}

\subsection*{Test 3}           
\begin{equation}
  h(x) = \frac{\exp(x^{2.5})}{\log(x) + \coth(x)} 
\end{equation}

\subsection*{Test 4}           
\begin{equation}
  k(x) = \frac{5\tan(x^{2.5})}{x^3 + 25x + 9} 
\end{equation}

\subsection*{Speed Tests}

When the test functions were implemented at run time vs directly implemented and run $10^7$ times, the following times were recorded. As you can see the direct implementations were always faster than the run time implementations.

\begin{table}[h]
  \centering 
  \begin{tabular}{c|ccc}
           & $t_\textrm{virtual}$  & $t_\textrm{direct}$ & $t_\textrm{direct}/t_\textrm{virtual}$  \\
    \hline
    $f(x)$ & 0.659322s & 0.592783s & 1.112248 \\
    $g(x)$ & 1.27494s  & 0.622899s & 2.046784 \\
    $h(x)$ & 9.66068s  & 7.41847s  & 1.302247 \\
    $k(x)$ & 7.23954s  & 4.96945s  & 1.456809
  \end{tabular}
\end{table}

\end{document}
